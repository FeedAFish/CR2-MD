% Options for packages loaded elsewhere
\PassOptionsToPackage{unicode}{hyperref}
\PassOptionsToPackage{hyphens}{url}
%
\documentclass[
  11pt,
  dvipsnames]{article}
\usepackage{amsmath,amssymb}
\usepackage{lmodern}
\usepackage{iftex}
\ifPDFTeX
  \usepackage[T1]{fontenc}
  \usepackage[utf8]{inputenc}
  \usepackage{textcomp} % provide euro and other symbols
\else % if luatex or xetex
  \usepackage{unicode-math}
  \defaultfontfeatures{Scale=MatchLowercase}
  \defaultfontfeatures[\rmfamily]{Ligatures=TeX,Scale=1}
\fi
% Use upquote if available, for straight quotes in verbatim environments
\IfFileExists{upquote.sty}{\usepackage{upquote}}{}
\IfFileExists{microtype.sty}{% use microtype if available
  \usepackage[]{microtype}
  \UseMicrotypeSet[protrusion]{basicmath} % disable protrusion for tt fonts
}{}
\makeatletter
\@ifundefined{KOMAClassName}{% if non-KOMA class
  \IfFileExists{parskip.sty}{%
    \usepackage{parskip}
  }{% else
    \setlength{\parindent}{0pt}
    \setlength{\parskip}{6pt plus 2pt minus 1pt}}
}{% if KOMA class
  \KOMAoptions{parskip=half}}
\makeatother
\usepackage{xcolor}
\IfFileExists{xurl.sty}{\usepackage{xurl}}{} % add URL line breaks if available
\IfFileExists{bookmark.sty}{\usepackage{bookmark}}{\usepackage{hyperref}}
\hypersetup{
  hidelinks,
  pdfcreator={LaTeX via pandoc}}
\urlstyle{same} % disable monospaced font for URLs
\usepackage[margin=1in]{geometry}
\usepackage{listings}
\newcommand{\passthrough}[1]{#1}
\lstset{defaultdialect=[5.3]Lua}
\lstset{defaultdialect=[x86masm]Assembler}
\usepackage{longtable,booktabs,array}
\usepackage{calc} % for calculating minipage widths
% Correct order of tables after \paragraph or \subparagraph
\usepackage{etoolbox}
\makeatletter
\patchcmd\longtable{\par}{\if@noskipsec\mbox{}\fi\par}{}{}
\makeatother
% Allow footnotes in longtable head/foot
\IfFileExists{footnotehyper.sty}{\usepackage{footnotehyper}}{\usepackage{footnote}}
\makesavenoteenv{longtable}
\usepackage{graphicx}
\makeatletter
\def\maxwidth{\ifdim\Gin@nat@width>\linewidth\linewidth\else\Gin@nat@width\fi}
\def\maxheight{\ifdim\Gin@nat@height>\textheight\textheight\else\Gin@nat@height\fi}
\makeatother
% Scale images if necessary, so that they will not overflow the page
% margins by default, and it is still possible to overwrite the defaults
% using explicit options in \includegraphics[width, height, ...]{}
\setkeys{Gin}{width=\maxwidth,height=\maxheight,keepaspectratio}
% Set default figure placement to htbp
\makeatletter
\def\fps@figure{htbp}
\makeatother
\setlength{\emergencystretch}{3em} % prevent overfull lines
\providecommand{\tightlist}{%
  \setlength{\itemsep}{0pt}\setlength{\parskip}{0pt}}
\setcounter{secnumdepth}{5}
\usepackage{setspace}
\usepackage{float}
\usepackage{fontspec}
\usepackage{subfig}
\usepackage[french]{babel}
\usepackage{csquotes}
\setmonofont{JetBrains Mono}[Contextuals=Alternate]
\floatplacement{figure}{H}
\lstset{ 
  language=python,                     % the language of the code
  basicstyle=\small\ttfamily, % the size of the fonts that are used for the code
  stepnumber=1,                   % the step between two line-numbers. If it is 1, each line
                                  % will be numbered
  numbersep=5pt,                  % how far the line-numbers are from the code
  backgroundcolor=\color{cyan!5},  % choose the background color. You must add \usepackage{color}
  showspaces=false,               % show spaces adding particular underscores
  showstringspaces=false,         % underline spaces within strings
  showtabs=false,                 % show tabs within strings adding particular underscores
  frame=single,                   % adds a frame around the code
  rulecolor=\color{black},        % if not set, the frame-color may be changed on line-breaks within not-black text (e.g. commens (green here))
  tabsize=2,                      % sets default tabsize to 2 spaces
  captionpos=b,                   % sets the caption-position to bottom
  breaklines=true,                % sets automatic line breaking
  breakatwhitespace=false,        % sets if automatic breaks should only happen at whitespace
  keywordstyle=\color{RoyalBlue},      % keyword style
  commentstyle=\color{Green},   % comment style
  stringstyle=\color{Orange},      % string literal style
}
\makeatletter
\renewcommand\paragraph{\@startsection{paragraph}{4}{\z@}%
        {-2.5ex\@plus -1ex \@minus -.25ex}%
        {1.25ex \@plus .25ex}%
        {\normalfont\normalsize\bfseries}}
\makeatother
\setcounter{secnumdepth}{4}
\hypersetup{
    colorlinks = true,
}
\ifLuaTeX
  \usepackage{selnolig}  % disable illegal ligatures
\fi

\author{}
\date{\vspace{-2.5em}}

\begin{document}

\onehalfspacing

\pagenumbering{gobble}

\begin{titlepage}
\vspace*{\fill}
\begin{center}
\LARGE{\textbf{COMPTE RENDU}}\\

\Large{\textbf{Méthodes de différences finies et méthodes de Monte-Carlo}}\\
\Large{\textbf{pour l’équation de la chaleur}}\\
\vspace*{1\baselineskip}
\Large{\textbf{Membres}}\\
PHAM Tuan Kiet\\
VO Van Nghia\\
\vfill % equivalent to \vspace{\fill}
\vspace*{\fill}
\today
\end{center}
\end{titlepage}

\newpage

\newpage

\newpage
\pagenumbering{arabic}

{
\setcounter{tocdepth}{3}
\tableofcontents
}
\newpage

\hypertarget{pruxe9sentation-du-probluxe8me}{%
\section{Présentation du problème}\label{pruxe9sentation-du-probluxe8me}}

L'objectif de ce TP est d'étudier la résolution de l'équation de la chaleur en comparant la méthode des différences finies et la méthode de Monte-Carlo. Soit \(L = 1\), \(\Omega = ]0,L[ \times ]0,L[\) et \(T\) un réel strictement positif. Le problème è résoudre est le suivant :
\[\frac{\partial u}{\partial t} + V \cdot \nabla u - D \Delta u = f\] sur \([0,T] \times \Omega\),
avec la condition initiale :
\[u(0,x,y)=0, \quad \forall (x,y) \in \Omega\]
et les condition aux limites :

\begin{itemize}
\tightlist
\item
  en \(x \in \{0,L\}, \quad \forall (t,y) \in [0,T]\times [0,L], \quad \frac{\partial u}{\partial x}(t,x,y)=0\)
\item
  en \(y \in \{0,L\}, \quad \forall (t,y) \in [0,T]\times [0,L], \quad u(t,x,0)=0, \quad u(t,x,L)=1\)
\end{itemize}

Le coefficient \(D=0.2\), strictement positif, correspond à la diffusivité thermique du fluide. \(V = (V_1,V_2)\) le champ de vitesse du fluide avec:
\[ V = (V_1(x,y),V_2(x,y)) = V_0 (-sin(\frac{\pi x}{L})cos(\frac{\pi y}{L}),sin(\frac{\pi y}{L})cos(\frac{\pi x}{L}))\]
où \(V_0 = 1\). La fonction \(f\) correspond à la source de chaleur avec l'expression:
\[\forall(t,x,y) \in [0,T] \times \Omega, \quad f(t,x,y)=256(\frac{x}{L})^2 (1-\frac{x}{L})^2(\frac{y}{L})^2 (1-\frac{y}{L})^2\]

\hypertarget{muxe9thode-des-diffuxe9rences-finies}{%
\section{Méthode des différences finies}\label{muxe9thode-des-diffuxe9rences-finies}}

Dans cette première partie, on s'intéresse à la résolution par différences finies. Soit \(K\) un entier strictement positif. Les sommets de la grille sont par définition les \((K + 1)^2\) points \(X_{i,j}\) de coordonnées \((ih,jh)\) avec \(h =\frac{L}{k}\) et \((i,j) \in \{0,\dots,K\} \times \{0,\dots,K\}\). On considère le schéma aux différences finies défini par les relations suivantes.

On note: \(V_{i,j}^1=V_1(X_{i,j}),V_{i,j}^2=V_2(X_{i,j})\) et \(f_{i,j}=f(X_{i,j})\).

\end{document}
